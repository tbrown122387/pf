\href{https://zenodo.org/badge/latestdoi/130237492}{\tt }

This is a template library for \href{https://en.wikipedia.org/wiki/Particle_filter}{\tt particle filtering}. Templated abstract base classes for different particle filters are provided (e.\+g. the Bootstrap Filter, the S\+I\+SR filter, the Auxiliary Particle Filter, the Rao-\/\+Blackwellized particle filter), as well as non-\/abstract (but indeed templated) base classes for closed-\/form filtering algorithms (e.\+g. Kalman Filter, Hidden Markov Model filter, etc.).

Once you have a certain model in mind, all you have to do is make it into a class that inherits from the filter you want to use.

\subsection*{Dependencies}

This code makes use of the following libraries\+:


\begin{DoxyItemize}
\item \href{http://eigen.tuxfamily.org/}{\tt Eigen v3.\+3}
\item \href{https://www.boost.org/}{\tt Boost v1.\+65.\+1}
\item \href{https://github.com/catchorg/Catch2}{\tt Catch2 v2.\+9.\+2}
\end{DoxyItemize}

Also, your compiler must enable C++17.

\subsection*{Installation}

\subsubsection*{Option 1\+: Install from Github}

{\ttfamily git clone} this Github repostory, {\ttfamily cd} into the directory where everything is saved, then run the following commands\+: \begin{DoxyVerb}mkdir build && cd build/
cmake .. -DCMAKE_INSTALL_PREFIX:PATH=/usr/local
sudo cmake --build . --config Release --target install --parallel
\end{DoxyVerb}


You may subsitute another directory for {\ttfamily /usr/local}, if you wish. This will also build unit tests that can be run with the following command (assuming you\textquotesingle{}re still in {\ttfamily build/})\+: \begin{DoxyVerb}./test/pf_test
\end{DoxyVerb}


\subsubsection*{Option 2\+: Drag-\/and-\/drop {\ttfamily .h} files}

This is a header-\/only library, so there will be no extra building necessary. If you just want to copy the desired header files from {\ttfamily include/pf} into your own project, and build that project by itself, that\textquotesingle{}s totally fine. There is no linking necessary, either. If you go this route, though, make sure to compile with C++17 enabled. Note, also, that this code all makes use of \href{http://eigen.tuxfamily.org/}{\tt Eigen v3.\+3} and \href{https://www.boost.org/}{\tt Boost v1.\+65.\+1}. Unit tests use the \href{https://github.com/catchorg/Catch2}{\tt Catch2 v2.\+9.\+2} library.

\subsection*{Examples}

Don\textquotesingle{}t know how to use this? No problem. Check out the \href{https://github.com/tbrown122387/pf/tree/master/examples}{\tt {\ttfamily examples}} sub-\/directory. This is a stand-\/alone cmake project, so you can just copy this sub-\/directory anywhere you like, and start editing.

For example, copy to {\ttfamily Desktop} and have at it\+: \begin{DoxyVerb}cp -r ~/pf/examples/ ~/Desktop/
cd Desktop/examples/
mkdir build && cd build
cmake ..
make
\end{DoxyVerb}


\subsection*{Paper}

A full-\/length tutorial paper is available \href{https://arxiv.org/abs/2001.10451}{\tt here.}

\subsection*{Citation}

Click the \char`\"{}\+D\+O\+I\char`\"{} link above. Or, if you\textquotesingle{}re impatient, click \href{https://zenodo.org/record/2633289/export/hx}{\tt \textquotesingle{}here\textquotesingle{}} for a Bibtex citation. 